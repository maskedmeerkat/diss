% !TeX root = ../main.tex
{\newgeometry{top=1.5cm,bottom=1.5cm,right=1.5cm,left=1.5cm}

\begin{titlepage}
  \centering
  \begin{minipage}[t]{9.8cm}
      \begin{center}
          {\large \textbf{Deep Inverse Sensor Models as Priors for evidential Occupancy Mapping}}
          \bigskip
      \end{center}
  \end{minipage}
  \begin{minipage}[t]{9.8cm}
      \begin{center}
          {\large \textcolor{ikagrey}{\textit{\textbf{Tiefe inverse Sensormodelle als A-priori-Information in evidenzbasierten Belegungskarten}}}}
          \bigskip
      \end{center}
  \end{minipage}
  \vfill
  \begin{minipage}[t]{16.5cm}
      \begin{center}
          Von der Fakultät für Maschinenwesen der Rheinisch-Westfälischen Technischen Hochschule Aachen zur Erlangung des akademischen Grades eines Doktors der Ingenieurwissenschaften genehmigte Dissertation
      \end{center}
  \end{minipage}
  % \vspace{2cm}
  \vfill
  \begin{minipage}[t]{9.8cm}
      \begin{center}
      	vorgelegt von\\
      	\vspace{\baselineskip}
          Daniel Max Bauer
      \end{center}
  \end{minipage}
\vfill
\begin{minipage}[t]{16.5cm}
	\raggedright
	Berichter: Univ.-Prof. Dr.-Ing Lutz Eckstein\\
	\phantom{Berichter:} Univ.-Prof. Dr. sc. techn. Bastian Leibe\\ 
	\vspace{\baselineskip}
	Tag der mündlichen Prüfung: 14.02.2024\\
	\vspace{\baselineskip}
	,,Diese Dissertation is auf den Internetseiten der Universitätsbibliothek online verfügbar.''
\end{minipage}
\vfill
\end{titlepage}
\restoregeometry
}
