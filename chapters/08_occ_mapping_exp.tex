% !TeX root = ../main.tex
\chapter{Deep, evidential ISMs as Priors for Occupancy Mapping Experiments}
\label{ch:deep_ev_isms_as_prior_for_occmaps_exp}
%==========================================================================%
%
%==========================================================================%
\section{Experiment to verify Choices of Combination Rules}
\label{sec:exp_choice_comb_rule}
In this section, a qualitative verification of the combination rules' properties derived in Sec. \ref{subsec:method_to_use_deep_isms_as_priors_in_occmaps} is provided. To demonstrate the properties, an evidential occupancy signal (first row Fig. \ref{fig:ds_comb_rule_comparison}) is fused over time with a mass initialized with $m_u = 1$ using different combination rules. The signal starts with a step-wise ramp-up of the occupied mass, followed by a completely contradictory signal, a signal with partial conflict, the consecutive contradicting signals and, eventually, a signal with full conflict.
\begin{figure}[H]
	\begin{center}
		\import{imgs/08_occ_mapping_exp/choice_of_comb_rule}{ds_comb_rule_comparison.pgf}
		\caption{\label{fig:ds_comb_rule_comparison}This figure shows the qualitative evaluation of the different evidential combination rules given input signals for free, occupied and unknown mass (first subplot) over time with an exemplary lower unknown threshold $\underline{m}_u$ overlayed.}
	\end{center}
\end{figure} 
Beginning with Dempster's rule, it can be seen that the fused mass always fully converges to the class of the signal with the highest portion (see Dempster graph e.g. in the time interval $[0, 20]$). Given temporal independent information and no conflict, this behavior would be desired since each time step provides new evidence of the state. However, in presence of conflict, the fused mass should converge to a state representing the portion of conflict to represent cells being dynamic. Since Dempster's rule lacks the ability to do so, it is disqualified for usage in this work.

Similar to Dempster's rule, Yager's rule, in absence of conflict, converges fully to the dominating class (see Yager graph e.g. in the time interval $[0, 20]$). In presence of conflicting mass, Yager's rule recuperates unknown mass (see e.g. time interval $[35, 45]$) and allows to directly switch the state, as opposed to Dempster's rule (compare Dempster and Yager graph in time interval $[35, 45]$). The recuperation capability is useful in the initialization phase to correct for state changes without falling below $\underline{m}_u$. Also, while Yager's rule is capable of representing the conflict state of a signal (see Yager graph in the time interval $[80, 100]$ and $[140, 160]$) it never reaches the original input conflict due to the recuperated unknown mass. It shall be argued that the property of recuperating unknown mass outweighs the deficit of biased conflict representation for the purpose of initializing the occupancy state.  

To solve the biased conflict state in the convergence phase, the YaDer rule has been proposed. It can be seen that by equally distributing the conflict into the conflicting classes instead of shifting it to the unknown class, the fused mass converges to the input's conflict state (see YaDer graph in the time interval $[80, 100]$ and $[140, 160]$). Also, in case of absent conflict, YaDer rule, like Yager and Dempster, fully converges to the dominating class (see YaDer graph e.g. in the time interval $[0,20]$). Thus, the YaDer rule combines both the Yager rule's property to represent conflict and Dempster rule's property to strictly reduce the unknown mass, giving it its name. These properties makes YaDer rule an ideal candidate to be used in the convergence phase.

Eventually, the lower-bounded (lb) Yager rule's property to only converge to the input signal until the same unknown mass is reached is shown in the last graph (e.g. in time interval $[0, 30]$). It can be seen that there is a discrepancy between the converged fused state and the input mass. This is due to the fact that the input is assumed to originate from a deep \gls{ism} with limited certainty. Therefore, the input is first rescaled into the interval $m_u \in [\underline{m}_u, 1]$ before fed into the lb Yager rule. Moreover, it can be seen when comparing the time intervals $[35, 45]$ and $[45, 55]$ that a state switch between free and occupied can only be realized if the conflicting signal has high enough certainty relative to the current state. This is desired in order not to overwrite a highly certain estimate, potentially predicted close to data, with lower certain ones, potentially based on extrapolations. Problematic, however, is that the state change is not fully completed letting the fused mass converge to a conflicting state. This is due to increasingly discounting the input signal when reducing the difference in unknown mass between input and fused signal. This property ensures that once $\underline{m}_u$ is reached, the deep \gls{ism} can not further influence the fused mass (see lower-bounded Yager graph in the time interval $[100, 160]$). Here, it shall argued that the benefit of reducing the influence of outliers and the property of deactivating the deep \gls{ism}'s influence outweigh the shortcoming of incomplete state change for the initialization phase. 
%==========================================================================%
%
%==========================================================================%
\section{Analysis of redundant Information in Deep ISMs}
\label{sec:exp_analyze_redundant_info}
%=================================%
%
%=================================%
\subsection{Setup of Redundancy Analysis in Deep ISMs}
\label{subsec:setup_of_red_analy}
To verify \todo{Hypothesis bla}, occupancy maps are created in four ways. The baseline is the direct accumulation using Yager's rule, abbreviated as "accum.". This is evaluated for the deep as well as the geo ISM. Second, a naive solution is evaluated which moves all free and occupied mass to unknown for predictions with $m_u>=0.9$ to hinder small predictions to accumulate over time ("accum. with cutoff"). Third, the method as proposed in \todo{section or equation} is used to reduce the redundancy ("accum. with redundancy reduction"). Eventually, the fusion using the deep and geo ISMs by directly accumulating the predictions with Yager's rule is evaluated ("accum. fusion"). Here, the fusion is investigated to analyze whether the deep ISM indeed overwrites most of the geo ISM's predictions as proposed in \todo{Hypothesis bla}.
%
The evaluation is performed using only the accumulated radar detections of the recent 20 timesteps (R$_{20}$) since the focus of this work lies on radar occupancy mapping. Additionally, it is assumed that the effects are similar using other sensor modalities. The geo ISM method is used as described in \todo{ref to geo ISM} and ShiftNet, as described \todo{ref to ShiftNet}, is used as a deep ISM.
%
The metrics used are the normed confusion matrix \todo{ref to conf matrix definition} and the mIoU \todo{ref to mIoU definition}. Here, the metrics are evaluated separately in all of the area in a 20m vicinity around the ego vehicle trajectory ("whole mapped area") and in an area of 15 pixels around occupied ground truth pixels in the reference occupancy maps. Here, the mIoU is only evaluated around the occupied borders to quantify the cleanliness. The reference maps are created by accumulating the geo ILM. Two examples of the evaluation areas can be found in \ref{fig:qual_analysis_of_redundant_info}. The scenes mapped are solely taken from the test set which was not used during training.
%=================================%
%
%=================================%
\subsection{Results of Redundancy Analysis in Deep ISMs}
\label{subsec:results_of_red_analy}
\begin{itemize}
	\item show alleys for both mapping methods
	\item show mapping progress that illustrates the overwriting of correct assignments 
\end{itemize}
\begin{figure}[H]
	\begin{center}
		\begin{tabular}{c|c|ccc|ccc}
			& \backslashbox{}{\scriptsize{$k$}} & $f$ & $o$ & $u$ & $f$ & $o$ & $u$\\
			\hline
			geo IR$_{20}$M &$p(k|f)$ & \textcolor{mygreen}{57.9} & \textcolor{myred}{2.3} & 39.8& \textcolor{mygreen}{44.3} & \textcolor{myred}{5.9} & 49.8 \\
			accumulated &$p(k|o)$ & \textcolor{myred}{16.1} & \textcolor{mygreen}{36.3} & 47.6& \textcolor{myred}{15.9} & \textcolor{mygreen}{36.0} & 48.1 \\
			&$p(k|u)$ & 2.8 & 5.7 & 91.5& 4.0 & 9.7 & 86.3 \\
			& mIoU & - & - & - &16.8&24.6&14.2 \\
			\hline	
			deep IR$_{20}$M &$p(k|f)$ & \textcolor{mygreen}{86.6} & \textcolor{myred}{11.7} & 1.7& \textcolor{mygreen}{68.2} & \textcolor{myred}{28.4} & 3.4 \\
			accumulated &$p(k|o)$ & \textcolor{myred}{27.2} & \textcolor{mygreen}{69.6} & 3.2& \textcolor{myred}{27.3} & \textcolor{mygreen}{69.4} & 3.3 \\
			&$p(k|u)$ & 28.9 & 57.5 & 13.6& 27.2 & 64.5 & 8.3 \\
			& mIoU & - & - & - &15.3&12.4&2.8 \\
			\hline
			deep IR$_{20}$M &$p(k|f)$ & \textcolor{mygreen}{87.4} & \textcolor{myred}{10.5} & 2.1& \textcolor{mygreen}{69.4} & \textcolor{myred}{26.7} & 3.9 \\
			accumulated with&$p(k|o)$ & \textcolor{myred}{28.3} & \textcolor{mygreen}{67.0} & 4.7& \textcolor{myred}{28.4} & \textcolor{mygreen}{66.8} & 4.8 \\
			cutoff &$p(k|u)$ & 33.7 & 34.7 & 31.6& 30.4 & 51.9 & 17.7 \\
			& mIoU & - & - & - &15.1&16.5&5.2 \\
			\hline
			deep \& geo IR$_{20}$M &$p(k|f)$ & \textcolor{mygreen}{86.7} & \textcolor{myred}{11.6} & 1.7& \textcolor{mygreen}{68.4} & \textcolor{myred}{28.3} & 3.3 \\
			accumulated &$p(k|o)$ & \textcolor{myred}{27.3} & \textcolor{mygreen}{69.5} & 3.2& \textcolor{myred}{27.4} & \textcolor{mygreen}{69.3} & 3.3 \\
			fusion&$p(k|u)$ & 29.0 & 57.4 & 13.6& 27.3 & 64.4 & 8.3 \\
			& mIoU & - & - & - &15.3&12.4&2.7 \\
			\hline
			deep IR$_{20}$M &$p(k|f)$ & \textcolor{mygreen}{88.5} & \textcolor{myred}{5.9} & 5.6& \textcolor{mygreen}{71.6} & \textcolor{myred}{16.7} & 11.7 \\
			accumulated with &$p(k|o)$ & \textcolor{myred}{29.0} & \textcolor{mygreen}{55.7} & 15.3& \textcolor{myred}{29.0} & \textcolor{mygreen}{55.4} & 15.6 \\
			redundancy reduction &$p(k|u)$ & 26.3 & 12.9 & 60.8& 26.3 & 23.4 & 50.3 \\
			& mIoU & - & - & - &16.8&23.4&11.8 \\
			\hline
			& & \multicolumn{3}{c|}{\scriptsize{whole mapped area}} & \multicolumn{3}{c}{\scriptsize{boundary area}}
		\end{tabular}
		\caption{\label{tab:conf_mat_redunt_info}Normed confusion matrix for the three mapping variants using deep ISMs based on ShiftNet applied on different sensor modalities. See \ref{subsec:setup_of_red_analy} for further information on the methods and abbreviations used.}
	\end{center}
\end{figure}
Fig. \ref{tab:conf_mat_redunt_info} shows that, even though, the geo IR$_{20}$M's occupancy map has the least true positive rates overall, it also has by far the least false rates. This is most significant when comparing its occupied false rate with the deep IR$_{20}$M maps. Here, the deep IR$_{20}$Ms without redundancy reduction produce about five times and the one with the reduction about twice the amount of false occupied predictions in the whole mapped area.

%When comparing the results in Fig. \ref{tab:conf_mat_redunt_info} it can be seen that in general, the metrics improve from top to bottom based on their respective inputs. Comparing the methods used to map, accumulating the predictions with Yager's rule leads overall to the highest true positive and false negative rates and, thus, lowest unknown distribution. 

%Looking at the variant with cutoff, it can be seen that overall more unknown mass is being kept in comparison to the standard accumulation, due to shifting small predictions to unknown. This approach overall reduces both false and true occupancy rates in about equal portions compared to the standard accumulation. At the same time, it increases the rates for free space also at about the same level. However, when looking at the false rates of free and occupied in unknown areas, both are overall being reduced with the reduction in occupied rate being more significant. This shows two things. First, there are in general more low probability predictions for the occupied class. This is why the influence of occupied predictions is overall reduced while free predictions become more prominent. Second, in areas marked as unknown in the ground truth, which are most likely occluded, both free and occupied predictions are overall smaller.

%Next, for the redundancy reduced accumulation, the overall occupied and free rates, both true and false, are significantly smaller compared to both afore analyzed methods. This is to be expected since the accumulation of predictions is thresholded by the certainty of the predictions themselves. To be more specific, the reduction in absolute numbers is overall about equal for free predictions compared to the standard accumulation method. For occupied predictions, the absolute reduction in true rate is bigger than the false rate reduction. In relative terms, however, the positive rates are less influenced than the false rates making the occupied predictions more reliable as compared to the standard accumulation. This improvement is shown more clearly when looking at the resulting mIoU scores for the three methods in Fig. \ref{tab:miou_redunt_info}. It can be seen that the redundancy reduced accumulation consistently outperforms the other mapping variants. 
%
\begin{figure}[H]
	\begin{center}
		\import{imgs/08_occ_mapping_exp/analysis_of_redundant_info}{analysis_of_redundant_info_scenes_84_n_260.pdf_tex}
		\caption{\label{fig:qual_analysis_of_redundant_info}Qualitative results of the three mapping approaches using different sensor modalities for two scenes. Here, the first row shows the ground truth map created by accumulating the geo ILM with the evaluation area overlayed in white around the occupied pixels. The other maps are created using deep ISMs based on ShiftNet with there respective sensor inputs (see Sec. \ref{subsec:setup_of_red_analy}).}
	\end{center}
\end{figure}

\textbf{Discussion}
%==========================================================================%
%
%==========================================================================%
\section{Analysis of Deep ISM Priors for Occupancy Mapping}
\label{sec:exp_analyze_prior_properties}
\textbf{Setup}

\textbf{Experiment}
\begin{figure}[H]
\begin{center}
\begin{tabular}{c|c|ccc|ccc|ccc}
	&\backslashbox{}{\scriptsize{$k$}} & $f$ & $o$ & $u$ & $f$ & $o$ & $u$ & $f$ & $o$ & $u$\\
\hline
	geo IR$_{20}$M &$p(k|f)$ & \textcolor{mygreen}{59.9} & \textcolor{myred}{8.1} & 32.0& \textcolor{mygreen}{100.0} & \textcolor{myred}{0.0} & 0.0& \textcolor{mygreen}{0.0} & \textcolor{myred}{20.9} & 79.1 \\
	accumulated &$p(k|o)$ & \textcolor{myred}{21.7} & \textcolor{mygreen}{50.1} & 28.2& \textcolor{myred}{0.0} & \textcolor{mygreen}{100.0} & 0.0& \textcolor{myred}{42.7} & \textcolor{mygreen}{0.0} & 57.3 \\
	&$p(k|u)$ & 11.1 & 33.7 & 55.2& 0.0 & 0.0 & 100.0& 25.6 & 74.4 & 0.0 \\
\hline
	deep \& geo IR$_{20}$M &$p(k|f)$ & \textcolor{mygreen}{78.3} & \textcolor{myred}{16.7} & 5.0& \textcolor{mygreen}{88.1} & \textcolor{myred}{9.3} & 2.6& \textcolor{mygreen}{63.1} & \textcolor{myred}{28.4} & 8.5 \\
	accumulated &$p(k|o)$ & \textcolor{myred}{29.2} & \textcolor{mygreen}{65.1} & 5.7& \textcolor{myred}{8.3} & \textcolor{mygreen}{89.8} & 1.9& \textcolor{myred}{47.9} & \textcolor{mygreen}{42.4} & 9.7 \\
	fusion &$p(k|u)$ & 33.4 & 48.9 & 17.7& 41.0 & 29.8 & 29.2& 23.6 & 73.2 & 3.2 \\
\hline
	deep \& geo IR$_{20}$M &$p(k|f)$ & \textcolor{mygreen}{77.8} & \textcolor{myred}{13.0} & 9.2& \textcolor{mygreen}{91.4} & \textcolor{myred}{6.0} & 2.6& \textcolor{mygreen}{57.0} & \textcolor{myred}{24.1} & 18.9 \\
	accumulated &$p(k|o)$ & \textcolor{myred}{26.2} & \textcolor{mygreen}{63.5} & 10.3& \textcolor{myred}{5.2} & \textcolor{mygreen}{92.5} & 2.3& \textcolor{myred}{47.8} & \textcolor{mygreen}{33.2} & 19.0 \\
	...&$p(k|u)$ & 27.8 & 43.0 & 29.2& 31.5 & 18.8 & 49.7& 23.2 & 73.2 & 3.6 \\
\hline
	\multicolumn{2}{c|}{\textbf{ShiftNet}} & \multicolumn{3}{c|}{\scriptsize{geo IR$_{20}$M$(m_u) < 1$}} & \multicolumn{3}{c|}{\scriptsize{geo IR$_{20}$M$(m_u) < 1$ \& correct}} & \multicolumn{3}{c}{\scriptsize{geo IR$_{20}$M$(m_u) < 1$ \& false}} 
\end{tabular}
\caption{\label{fig:quant_analysis_of_prior}}
\end{center}
\end{figure}
The table distinguishes between areas touched and untouched ($m_u = 1$) by the original geo IRM which where still inside an area of 20m around the ego vehicles trajectory (see \todo{Fig} for an example of the mapped area). Thus, the majority of the so called untouched areas lies at the borders of the mapped area. 
\begin{figure}[H]
	\begin{center}
	\begin{tabular}{c|ccc}
		& \scriptsize{deep IR$_{20}$M} & \scriptsize{deep IDM} & \scriptsize{deep IDR$_{20}$M}\\
		\hline
		\scriptsize{p(fused Map(m$_u$ < 0.300) | geo IRM Map(m$_u$ = 1), $\underline{m}_u$=0.3)} & \scriptsize{0.00023\%} & \scriptsize{0.00026\%} & \scriptsize{0.00043\%}\\
		\scriptsize{\#(fused Map(m$_u$ < 0.300) | geo IRM Map(m$_u$ = 1), $\underline{m}_u$=0.3)} & \scriptsize{2476} & \scriptsize{2735} & \scriptsize{4613}\\
		\scriptsize{\#(fused Map(m$_u$ < 0.298) | geo IRM Map(m$_u$ = 1), $\underline{m}_u$=0.3)} & \scriptsize{0} & \scriptsize{0} & \scriptsize{0}
	\end{tabular}
	\caption{\label{fig:prior_violations}Analysis of how often the lower bound on the unknown mass $\underline{m}_u$ has been violated in the fused map in areas where the geo IRM has not been active. The violations are given in percentage and absolute. Also, the amount of violations given some slack on the condition are provided.}
	\end{center}
\end{figure}

\begin{figure}[H]
	\begin{center}
		\import{imgs/08_occ_mapping_exp/analysis_of_prior}{analysis_of_prior_scenes_442_n_488.pdf_tex}
		\caption{\label{fig:qual_analysis_of_prior}Qualitative results of the three mapping approaches using different sensor modalities for two scenes. Here, the first row shows the ground truth map created by accumulating the geo ILM with the evaluation area overlayed in white around the occupied pixels. The other maps are created using deep ISMs based on ShiftNet with there respective sensor inputs (see Sec. \ref{subsec:setup_of_red_analy}).}
	\end{center}
\end{figure}
Qualitative results
\begin{itemize}
	\item upper white box in scene A: wrong extrapolation
	\item lower white box in scene A: shortcoming of geo IR$_{20}$M can be compensated by initializing those areas using the deep IR$_{20}$M
	\item both white boxes in scene B: errors in deep IR$_{20}$M don't converge to certainty (see image of initialized areas). Geo IR$_{20}$M keeps them in check... or something like that. 
\end{itemize}
\textbf{Discussion}
