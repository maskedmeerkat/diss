% !TeX root = ../main.tex
\chapter{Deep, evidential ISMs as Priors for Occupancy Mapping Experiments}
\label{ch:deep_ev_isms_as_prior_for_occmaps_exp}
%==========================================================================%
%
%==========================================================================%
\section{Experiment to verify Choices of Combination Rules}
\label{sec:exp_choice_comb_rule}
In this section, a qualitative verification of the combination rules' properties derived in Sec. \ref{subsec:method_to_use_deep_isms_as_priors_in_occmaps} is provided. To demonstrate the properties, an evidential occupancy signal (first row Fig. \ref{fig:ds_comb_rule_comparison}) is fused over time with a mass initialized with $m_u = 1$ using different combination rules. The signal starts with a step-wise ramp-up of the occupied mass, followed by a completely contradictory signal, a signal with partial conflict, the consecutive contradicting signals and, eventually, a signal with full conflict.
\begin{figure}[H]
	\begin{center}
		\import{imgs/08_occ_mapping_exp/choice_of_comb_rule}{ds_comb_rule_comparison.pgf}
		\caption{\label{fig:ds_comb_rule_comparison}This figure shows the qualitative evaluation of the different evidential combination rules given input signals for free, occupied and unknown mass (first subplot) over time with an exemplary lower unknown threshold $\underline{m}_u$ overlayed.}
	\end{center}
\end{figure} 
Beginning with Dempster's rule, it can be seen that the fused mass always fully converges to the class of the signal with the highest portion (see Dempster graph e.g. in the time interval $[0, 20]$). Given temporal independent information and no conflict, this behavior would be desired since each time step provides new evidence of the state. However, in presence of conflict, the fused mass should converge to a state representing the portion of conflict to represent cells being dynamic. Since Dempster's rule lacks the ability to do so, it is disqualified for usage in this work.

Similar to Dempster's rule, Yager's rule, in absence of conflict, converges fully to the dominating class (see Yager graph e.g. in the time interval $[0, 20]$). In presence of conflicting mass, Yager's rule recuperates unknown mass (see e.g. time interval $[35, 45]$) and allows to directly switch the state, as opposed to Dempster's rule (compare Dempster and Yager graph in time interval $[35, 45]$). The recuperation capability is useful in the initialization phase to correct for state changes without falling below $\underline{m}_u$. Also, while Yager's rule is capable of representing the conflict state of a signal (see Yager graph in the time interval $[80, 100]$ and $[140, 160]$) it never reaches the original input conflict due to the recuperated unknown mass. It shall be argued that the property of recuperating unknown mass outweighs the deficit of biased conflict representation for the purpose of initializing the occupancy state.  

To solve the biased conflict state in the convergence phase, the YaDer rule has been proposed. It can be seen that by equally distributing the conflict into the conflicting classes instead of shifting it to the unknown class, the fused mass converges to the input's conflict state (see YaDer graph in the time interval $[80, 100]$ and $[140, 160]$). Also, in case of absent conflict, YaDer rule, like Yager and Dempster, fully converges to the dominating class (see YaDer graph e.g. in the time interval $[0,20]$). Thus, the YaDer rule combines both the Yager rule's property to represent conflict and Dempster rule's property to strictly reduce the unknown mass, giving it its name. These properties makes YaDer rule an ideal candidate to be used in the convergence phase.

Eventually, the lower-bounded (lb) Yager rule's property to only converge to the input signal until the same unknown mass is reached is shown in the last graph (e.g. in time interval $[0, 30]$). It can be seen that there is a discrepancy between the converged fused state and the input mass. This is due to the fact that the input is assumed to originate from a deep \gls{ism} with limited certainty. Therefore, the input is first rescaled into the interval $m_u \in [\underline{m}_u, 1]$ before fed into the lb Yager rule. Moreover, it can be seen when comparing the time intervals $[35, 45]$ and $[45, 55]$ that a state switch between free and occupied can only be realized if the conflicting signal has high enough certainty relative to the current state. This is desired in order not to overwrite a highly certain estimate, potentially predicted close to data, with lower certain ones, potentially based on extrapolations. Problematic, however, is that the state change is not fully completed letting the fused mass converge to a conflicting state. This is due to increasingly discounting the input signal when reducing the difference in unknown mass between input and fused signal. This property ensures that once $\underline{m}_u$ is reached, the deep \gls{ism} can not further influence the fused mass (see lower-bounded Yager graph in the time interval $[100, 160]$). Here, it shall argued that the benefit of reducing the influence of outliers and the property of deactivating the deep \gls{ism}'s influence outweigh the shortcoming of incomplete state change for the initialization phase. 

