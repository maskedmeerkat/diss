% Präambel: Pakete laden etc.
% ===========================
%-----------------------Allgemein---------------------------%
%scrhack, um @addtolists warnings zu unterdrücken, bei Paketen, die KOMA kennt (insbesondere listings)
\usepackage{scrhack}
%Input und Output Zeichensatz
\usepackage[utf8]{inputenc}
\usepackage[T1]{fontenc}
%English
\usepackage[english]{babel}
%needed for babel
\usepackage{csquotes}
%change font or background color, e.g. in table headings
\usepackage{xcolor}
\usepackage{colortbl}
%hyperref. BEST PACKAGE EVER. See documentation. Set pdfborder={0 0 0} in final version
\usepackage[raiselinks, pdfpagelabels, hypertexnames=true, pdfborder={0 0 1}, hidelinks, citebordercolor=yellow, linktoc=all]{hyperref}
%Text in Dokument kann mehrspaltig sein
\usepackage{multicol}
%enables \textsubscript
% \usepackage{subscript} % No need for this packacge according to ctan.org
\usepackage{booktabs}

%enable \gls
\usepackage[automake,nonumberlist,acronym,toc,nopostdot,nomain,numberedsection=autolabel]{glossaries}
\newglossary[slg]{symbolslist}{syi}{syg}{Symbols} % create add. symbolslist
\makeglossaries
% Displays unit in symbole list
\newglossarystyle{symbunitlong}{
	\setglossarystyle{long4col}
	\renewcommand*{\glossentry}[2]{
		\glstarget{##1}{\glossentryname{##1}} %
		& \glssymbol{##1}
		& \glossentrydesc{##1}  \tabularnewline
	}
}
\makeindex
\setacronymstyle{long-short}
\glsaddall

%enable \todo
\usepackage[colorinlistoftodos]{todonotes}\reversemarginpar
%enable subfigures
\usepackage{subcaption}


%%%%%%%%%%%%%%%%%%%%%% ADDED BY MHO %%%%%%%%%%%%%%%%%%%%%%%%%%%


\usepackage{parskip}
\usepackage[titletoc]{appendix}
\usepackage{import}
% theorem styles (hypothesis, research questions, requirements,...)
\usepackage{ntheorem}
\theoremseparator{:}
\newtheorem{hyp}{Hypothesis}
\newtheorem{subhyp}{Hypothesis}
\renewcommand\thesubhyp{\thehyp\arabic{subhyp}}
\newtheorem{requ}{Research Question}
\newtheorem{subrequ}{Research Question}
\renewcommand\thesubrequ{\therequ\arabic{subrequ}}
\newtheorem{req}{Requirement}
\newtheorem{subreq}{Requirement}
\renewcommand\thesubreq{\thereq.\arabic{subreq}}
% to proprly include matplotlib graphs rendered as pgfs
%\usepackage{pgf}
\usepackage{pgfplots}
% new commands
\definecolor{mygreen}{rgb}{0.0, 0.5, 0.0}
\definecolor{myred}{rgb}{0.5, 0.0, 0.0}
\usepackage[linesnumbered,ruled,vlined]{algorithm2e}
\DontPrintSemicolon
% Define pseudocode formatting
\renewcommand{\KwSty}[1]{\textnormal{\textcolor{blue!90!black}{\ttfamily\bfseries #1}}\unskip}
\renewcommand{\ArgSty}[1]{\textnormal{\ttfamily #1}\unskip}
\SetKwComment{Comment}{\color{green!50!black}// }{}
\renewcommand{\CommentSty}[1]{\textnormal{\ttfamily\color{green!50!black}#1}\unskip}
\newcommand{\assign}{\leftarrow}
\newcommand{\var}{\texttt}
\newcommand{\FuncCall}[2]{\texttt{\bfseries #1(#2)}}
\SetKwProg{Function}{function}{}{}
\renewcommand{\ProgSty}[1]{\texttt{\bfseries #1}}



%-----------------------Formatierung---------------------------%
%set paper geometry
\usepackage[left=3cm,  % 3.3cm
    right=2cm, % 2.0cm
    top=3cm,
    bottom=2.5cm,
    footskip=1.75cm,
    headsep=1.75cm
    ]{geometry}

%set footer and header positions
\addtolength{\headsep}{-1\headheight}
\addtolength{\footskip}{-1\footheight}
%use scrlayer for custom header and footer
\usepackage[markcase=noupper,    % remove the uppercasing
                        headsepline=0.75pt
                     ]{scrlayer-scrpage}
%Define content of header and footer
\renewcommand{\chapterpagestyle}{scrheadings}
\renewcommand{\headfont}{\normalfont}
\pagestyle{scrheadings}
\clearscrheadfoot
\renewcommand{\chaptermark}[1]{\markboth{#1}{}}
\ihead{\makebox[1cm][l]{\thechapter}\leftmark}
\ohead{\thepage}
%font Arial
%\usepackage[scaled=1]{uarial}
\renewcommand{\rmdefault}{ptm} % Times new Roman
\renewcommand{\sfdefault}{phv} % Arial
\renewcommand{\familydefault}{\sfdefault} %select Arial
%define footnote format
\deffootnote%
{1.5em}% Einzug des Fußnotentextes; bei dreistelligen Fußnoten evtl. vergrößern
{1em}% zusätzlicher Absatzeinzug in der Fußnote
{%
\makebox[1.5em]% Raum für Fußnotenzeichen: ebenso groß wie Einzug des FN-Textes
[l]% Ausrichtung des Fußnotenzeichens: [r]echts, [l]inks
{\textsuperscript{\thefootnotemark}}% Fußnotenzeichen
}
% 1,25-facher Zeilenabstand
\usepackage{setspace}
\setstretch{1.166} %LaTeX has its own stretch at 11pt, which combined with this one gives 1.25
%Zeilenumbruch bei URLs im Literaturverzeichnis
\usepackage{ulem}
%Definiert, dass URL nach jedem Buchstaben gtrennt werden kann
\makeatletter
\g@addto@macro\UrlBreaks{\do\a\do\b\do\c\do\d\do\e\do\f\do\g\do\h\do\i%
\do\j\do\k\do\l\do\m\do\n\do\o\do\p\do\q\do\r\do\s\do\t\do\u\do\v\do\w%
\do\x\do\y\do\z\do\&\do\1\do\2\do\3\do\4\do\5\do\6\do\7\do\8\do\9\do\0}
\makeatother 
% Keine einzelnen Zeilen beim Anfang eines Abschnitts (Schusterjungen) 
\clubpenalty=10000 
% Keine einzelnen Zeilen am Ende eines Abschnitts (Hurenkinder) 
\widowpenalty= 10000 \displaywidowpenalty = 10000
%-----------------------MathMode---------------------------%
%serifenlose Formelschrift
%\usepackage{sansmath}\sansmath
%euro symbol
\usepackage{eurosym} 
%standard math package
\usepackage{amsmath}
\usepackage{mathtools}
%math symbols
\usepackage{amssymb}
%enables \mathscr for caligraphic writing
\usepackage{mathrsfs}
%-----------------------Kapitel---------------------------%
%11pt, bold
\setkomafont{chapter}{\bfseries\normalsize}
\setkomafont{section}{\bfseries\normalsize}
\setkomafont{subsection}{\bfseries\normalsize}
\setkomafont{subsubsection}{\bfseries\normalsize}
\setkomafont{paragraph}{\bfseries\normalsize}
%Indentation of chapter titles
\renewcommand{\chapterformat}{\makebox[1cm][l]{\thechapter}}
\RedeclareSectionCommand[
  %afterindent=false,
  beforeskip=0pt,
  afterskip=6pt]{chapter}
\renewcommand{\sectionformat}{\makebox[2cm][l]{\thesection}}
\RedeclareSectionCommand[
  %afterindent=false,
  beforeskip=6pt,
  afterskip=6pt]{section}
\renewcommand{\subsectionformat}{\makebox[2cm][l]{\thesubsection}}
\RedeclareSectionCommand[
  %afterindent=false,
  beforeskip=6pt,
  afterskip=6pt]{subsection}
\renewcommand{\subsubsectionformat}{\makebox[2cm][l]{\thesubsubsection}}
\RedeclareSectionCommand[
  %afterindent=false,
  beforeskip=6pt,
  afterskip=6pt]{subsubsection}
\renewcommand*{\paragraphformat}{\makebox[2cm][l]{\theparagraph}}
\RedeclareSectionCommand[
  %afterindent=false,
  beforeskip=6pt,
  afterskip=6pt]{paragraph}
%-----------------------Inhaltsverzeichnis---------------------------%
%Maximale Gliederungstiefe, die noch ins Inhaltsverzeichnis aufgenommen wird
%maximum depth for the table of contents
%Tiefe von 4 erlaubt, aber häufig Verwendung von 3
%\setcounter{tocdepth}{3}
\setcounter{tocdepth}{4}
%Nummerierung der Überschriften bis zu welcher Schachtelung
\setcounter{secnumdepth}{4}
% Formatierung des Inhaltsverzeichnisses
\setkomafont{disposition}{\normalfont} 
\addtokomafont{chapterentry}{\normalfont}
%COUNTER needed, to have 12pt space before AND after each new chapter, but 6pt else
%LaTeX lets us only set space before a contentsline, so we have to check, if a
%section is the first in its chapter, and only then add 12pt before the section
\newcounter{sectionref}
\setcounter{sectionref}{1}
%set indentation in table of contents
\makeatletter
\renewcommand{\l@chapter}{\vspace{12pt} \setcounter{sectionref}{1} \@dottedtocline{1}{0cm}{0.9cm}}
\renewcommand\l@section{ \ifthenelse{\value{sectionref}=1}{\vspace{12pt}}{\vspace{6pt}} \stepcounter{sectionref} \@dottedtocline{2}{0.3cm}{1.25cm}}
\renewcommand\l@subsection{\vspace{6pt}\@dottedtocline{3}{0.6cm}{1.34cm}}
\renewcommand*\l@subsubsection{\vspace{6pt}\@dottedtocline{4}{0.9cm}{1,82cm}}
%narrow dots, like in WORD
\renewcommand\@dotsep{1}% default is 4.5
\makeatother
%-----------------------Abbildungen---------------------------%
\usepackage{graphicx}
%import von .svg, .eps etc.
\usepackage{import}
\usepackage{epstopdf} %converting to PDF
%import graphic pictures depending on pdf or not
\usepackage{rotating} %rotate caption if in landscape

% used to draw arrows between table columns
\usepackage{tikz}
\usetikzlibrary{tikzmark}

%include pdf pages
\usepackage{pdfpages}
%enables eps graphic import
\usepackage{epstopdf}
%Enable float H option
\usepackage{float}
%set captions
\usepackage[format=hang,
    justification=justified,
    singlelinecheck=off,
    position=bottom,
    skip=6pt,
    belowskip=6pt
    ]{caption}
%enable subfigures inside a figure
%enables rotating figures without caption
\usepackage{lscape}
%let the caption begin at 2.5cm relative to textbox
\DeclareCaptionLabelFormat{ikalabel}{\makebox[2cm][l]{\figurename~\thefigure:}}
\captionsetup{labelformat=ikalabel,labelsep=none}
% Anpassung der Tabellen- und Bildbeschriftung an das IKA-Layout, e.g. "Abb. 1-1:    " (Teil 2)
\renewcommand{\thefigure}{{\thechapter-\arabic{figure}}}
\addto\captionsenglish{
\renewcommand{\figurename}{Fig.}
}
\renewcommand{\thesubfigure}{\roman{subfigure}}
%-----------------------Tabellen---------------------------%
%better tables
\usepackage{tabularx}
%enable long tables for list of abbreviations etc.
\usepackage{longtable}
%enable multirow or multicolumn table cells
\usepackage{multirow}
% Durchgehende Nummerierung für Abbildungen und Tabellen: IKA fordert Tabllen mit Abb-Caption: Abb. X-Y
\makeatletter
\let\c@table\c@figure
\makeatother
% Anpassung der Tabellen- und Bildbeschriftung an das IKA-Layout, e.g. "Abb. 1-1:    " (Teil 2)
\renewcommand{\thetable}{{\thechapter-\arabic{table}}}
\addto\captionsenglish{
\renewcommand{\tablename}{Fig.}
}
%Farben in Tabellen
\usepackage{colortbl}
%-----------------------Gleichungen---------------------------%
% Anpassung der Gleichungsbeschriftung an das IKA-Layout, e.g. "Gl. 1-1" bzw. Eq. im Englischen --> für Deutsch Eq. im Folgenden mit Gl. ersetzen
\makeatletter
%    \def\@eqnnum{{\normalfont \normalcolor Gl.~\theequation}}
    \def\tagform@#1{\maketag@@@{Eq.~\ignorespaces#1\unskip\@@italiccorr}}
\makeatother
% Anpassung der Gleichungsbeschriftung an das IKA-Layout, e.g. "Gl. 1-1:    " (Teil 2)
\renewcommand{\theequation}{{\thechapter-\arabic{equation}}}
%-----------------------Literaturverzeichnis---------------------------%
%maxbibnames = alle Autoren ausschreiben
\usepackage[maxalphanames=1, minalphanames=1, style=alphabetic, backend=biber, sorting = nyt, maxcitenames=5, maxbibnames=1000,giveninits=true]{biblatex}
\DeclareNameAlias{default}{family-given}

%zeilenumbruch nach autor
\renewcommand*{\mkbibnamefamily}[1]{\MakeUppercase{#1}}	% UpperCase für Namen
\renewcommand*{\newunitpunct}{\addcomma\space}   		% Komma statt Punkt
\renewcommand*{\finentrypunct}{}                        % kein Punkt am Ende der Einträge
\DeclareFieldFormat[article, book, inbook, inproceedings, misc, techreport, phdthesis]{citetitle}{#1}
\DeclareFieldFormat[article, book, inbook, inproceedings, misc, techreport, phdthesis]{title}{#1}

\renewcommand*{\labelnamepunct}{\newline}				% Zeilenumbruch statt Komma hinter Autoren
\renewcommand*{\labelalphaothers}{}						% "+" bei mehreren Autoren entfernen
\renewbibmacro{in:}{}									% "in:" Zusatz bei Artikel in Journaleinträgen entfernen
\renewcommand*{\bibfont}{}

% Formatierung der Zitierungslabels
\DeclareLabelalphaTemplate{
	\labelelement{
		\field[uppercase,final]{shorthand}
		\field[uppercase]{label}
		\field[uppercase,strwidth=3,strside=left,ifnames=1]{labelname}
		\field[uppercase,strwidth=1,strside=left]{labelname}
	}
	\labelelement{
		\field[strwidth=2,strside=right]{year}
	}
}
\renewcommand{\labelnamepunct}{\newline\bibsentence}
%zeilenumbruch nach titel
\usepackage{xpatch}
\makeatletter
\def\do#1{
  \ifcsdef{blx@bbx@#1}
    {\xpatchbibdriver{#1}
       {\printlist{language}%
        \newunit\newblock}
       {\printlist{language}%
        \printunit{\newline\bibsentence}}
       {}{}}
    {}} 
\abx@doentrytypes
\makeatother
%keine kursiven schriften im literaturverzeichnis
\DeclareFieldFormat{citetitle}{{#1}}
\DeclareFieldFormat{journaltitle}{{#1}}
\DeclareFieldFormat{issuetitle}{{#1}}
\DeclareFieldFormat{maintitle}{{#1}}
\DeclareFieldFormat{booktitle}{{#1}}
\DeclareFieldFormat{title}{{#1}}
%keine anführungsstriche um den titel
\DeclareFieldFormat[article]{title}{#1}
\DeclareFieldFormat[thesis]{title}{#1}
\DeclareFieldFormat[inproceedings]{title}{#1}
\DeclareFieldFormat[incollection]{title}{#1}
%kein in: vor journals
\renewbibmacro{in:}{} 
\urlstyle{same} %keine formatierung für urls
%commas betwwen mltiple cites
\renewcommand\multicitedelim{\addcomma\space}
%no "and" or "und" between authors
\DefineBibliographyExtras{english}{%
  \renewcommand*{\finalnamedelim}{\addcomma\addspace}%
}
%-----------------------Weitere Funktionalität---------------------------%
%Abkürzungsverzeichns
\usepackage{acronym}
%Listings
\usepackage{listings}
\lstset{
    frame=single,
    breaklines=true,
    basicstyle=\footnotesize\ttfamily, %monospace font
    escapeinside={(}{)}
}
%draw bezier curves and oter stuff
\usepackage{epic,bez123}
%use SI units
\usepackage[
    per-mode=symbol,
    mode=text,
    ]{siunitx}
%\sisetup{
%    range-phrase = ~bis~,
%    output-decimal-marker={,},
%}
\sisetup{
	detect-all,
	locale=DE
}

\DeclareSIUnit\none{\textendash}
% for lorem ipsum
\usepackage{lipsum}
%make length calculations
\usepackage{calc}

% RWTH Farbdefinitionen
\definecolor{blueRWTH}{cmyk}{1.0,0.5,0,0}
\definecolor{blueLightRWTH}{cmyk}{0.75,0.38,0,0}
\definecolor{blueLighterRWTH}{cmyk}{0.45,0.14,0,0}

\definecolor{greenRWTH}{cmyk}{0.7,0.0,1.0,0}
\definecolor{greenLightRWTH}{cmyk}{0.525,0,0.75,0}
\definecolor{greenLighterRWTH}{cmyk}{0.35,0,0.5,0}

\definecolor{orangeRWTH}{cmyk}{0,0.4,1.0,0}
\definecolor{orangeLightRWTH}{cmyk}{0,0.3,0.75,0}
\definecolor{orangeLighterRWTH}{cmyk}{0,0.2,0.5,0}

\definecolor{redRWTH}{cmyk}{0.15,1,1,0}
\definecolor{redLightRWTH}{cmyk}{0.1125,0.75,0.75,0}
\definecolor{redLighterRWTH}{cmyk}{0.075,0.5,0.5,0}
\definecolor{rwthLight}{cmyk}{0.0,0.0,0.0,0.5}
\definecolor{rwthLight2}{cmyk}{0.0,0.0,0.0,0.8}
\definecolor{yellowPaper}{cmyk}{0,0.02,0.15,0}

\definecolor{ikagrey}{RGB}{128,128,128}

% \pgfplotsset{
% 	colormap ={rwth}{color(0pt)=(white); color(1pt)=(blueRWTH); color(2pt)=(blueRWTH)}
% }

\usepackage{listofitems}

\newcommand{\drawtwodimensionalhistogram}[6]{%
	\setsepchar{,}
	\readlist\argx{#1}
	\edef\xmin{\argx[1]}
	\edef\xmax{\argx[2]}

	\readlist\argy{#2}
	\edef\ymin{\argy[1]}
	\edef\ymax{\argy[2]}
	
	\readlist\argp{#3}
	\edef\mux{\argp[1]}
	\edef\sigmax{\argp[2]}
	\edef\muy{\argp[3]}
	\edef\sigmay{\argp[4]}
	\edef\ellipseangle{\argp[5]}
	\edef\positiondata{#4}
	
	\readlist\argn{#5}
	\def\xlabelname{\argn[1]}
	\def\ylabelname{\argn[2]}
	\readlist\argu{#6}
	\def\xunit{\argu[1]}
	\def\yunit{\argu[2]}

	\input{content/figures/histogram.tex}
}
\setlength{\arrayrulewidth}{1pt}

\usepackage{collcell}
\usepackage{xstring}

\newcommand*{\MaxNumber}{1.0}%
\newcommand{\ApplyGradient}[1]{
\if\IfDecimal{#1}
	\pgfmathsetmacro{\PercentColor}{100.0*(abs(#1))/(\MaxNumber)}
	\edef\x{\noexpand\cellcolor{redRWTH!\PercentColor}}\x\textcolor{black}{#1}%
\else
	#1
\fi
}

\newcolumntype{R}{>{\collectcell\ApplyGradient}r<{\endcollectcell}}

\tikzstyle{io} = [trapezium, trapezium left angle=70, trapezium right angle=-70, align=center, text width = 2.8cm,minimum height=1.4cm, minimum width=2.8cm, draw=black, trapezium stretches=true,fill=orangeLighterRWTH!50]
\tikzstyle{process} = [rectangle, align=center, minimum width=3cm, text width=3cm, minimum height=1.4cm, draw=black,fill=blueLighterRWTH!50]
\tikzstyle{arrow} = [->, shorten >= 0.1cm, shorten <= 0.1cm]

%\let\vec\mathbf
\renewcommand\vec{\boldsymbol}
